\documentclass[12pt, letterpaper]{article}
\usepackage[utf8]{inputenc}
\usepackage{listings} % for code snippet
\usepackage{xcolor}

\definecolor{codegreen}{rgb}{0,0.6,0}
\definecolor{codegray}{rgb}{0.5,0.5,0.5}
\definecolor{codepurple}{rgb}{0.58,0,0.82}
\definecolor{backcolour}{rgb}{0.95,0.95,0.92}

\lstdefinestyle{mystyle}{
	backgroundcolor=\color{backcolour},   
	commentstyle=\color{codegreen},
	keywordstyle=\color{magenta},
	numberstyle=\tiny\color{codegray},
	stringstyle=\color{codepurple},
	basicstyle=\ttfamily\footnotesize,
	breakatwhitespace=false,         
	breaklines=true,                 
	captionpos=b,                    
	keepspaces=true,                 
	numbers=left,                    
	numbersep=5pt,                  
	showspaces=false,                
	showstringspaces=false,
	showtabs=false,                  
	tabsize=2
}

\lstset{style=mystyle}


\title{LKM PD 6}
\author{Muhammad Rizqi Ardiansyah}

\begin{document}
	
	\begin{titlepage}
		\maketitle
	\end{titlepage}
	
	\section{Instruksi Tugas}
	Tulislah sebuah flowchart dan pseudocode dari penyelesaian kasus dalam menampilkan semua bilangan genap yang terletak antara 20 sampai dengan 120 dengan menggunakan for!
	\section{Pseudocode}
	\begin{lstlisting}
for (i = 120; i <= 120; i++) do 
	if i % 2 == 0 do 
		print i
	\end{lstlisting}
	\section{Flowchart}
	
\end{document}